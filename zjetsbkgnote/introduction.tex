\section{Introduction}
\label{sec:intro}

The search of the Higgs boson is one of the primary goals of the ATLAS detector. Collisions at center of mass energies of $\sqrt{s}$ = 7 and 8 TeV yielded evidence for several decay modes of the Higgs and a combination between the ATLAS and CMS experimental results claims evidence for a new particle compatible with the Standard Model Higgs Boson decaying into taus. The first collisions at $\sqrt{s}=13$ TeV lead to the possibility of strengthening this evidence with a `pure' result derived from each detector independently. The analysis laid out by the ATLAS group features a 'cut based' analysis whereby the final state topology of the Higgs is exploited to enhance any new signals against their backgrounds. 

The primary irreducible background to Higgs searches with taus is the decay of the Z boson into a tau anti-tau pair. The analysis targets kinematic variables such as the resonant mass of the tau pair to try to remove as many Z type events whilst retaining as many Higgs type events as possible. This means that the remaining Z$\rightarrow\tau\tau$ events are in a highly contrived region of phase-space where the theoretical prediction of the events is highly important to the sensitivity of the analysis. 

In view of the complexity of the
relevant event properties, in the past the ATLAS collaboration endeavoured to rely as little as possible on simulation. However Z$\rightarrow\tau\tau$ model cannot be obtained directly from the collision data due to background contributions, e.g.  from events with other objects misidentified as tau decays. Events with two muons can be `embedded' with simulated tau decays such that kinematic quantities can be preserved. However such a process requires a large Z$\rightarrow\mu\mu$ data set and extensive validation.

In practice, particle physics analyses use Monte-Carlo (MC) generators to compare predictions from theory to data. An extensive system of simulation and reconstruction mimics the effects of the detector such that theoretical models can be compared directly with physics objects in data. A full description of all relevant processes in simulated MC is considered by many to be the only way that a process in the ATLAS detector can be observed. Moreover, it allows us to produce new events for testing and refining our analysis regardless of the performance of the LHC and ATLAS detector. As such the production of high statistic, high prescision MC modeling of the Z$\rightarrow\tau\tau$ process is key to the search for Higgs Bosons in the first $\sqrt{s}=13$TeV data with the ATLAS detector.

The Z$\rightarrow\tau\tau$ process in MC is highly complex. Due to the properties of the Z and the tau almost all observable quantities are correlated. To reduce Z and Drell-Yan (DY) contribution to the Higgs analysis the signal regions either have a high transverse mass (possibly with additional jets) or explicity have at least 2 additional jets. The production of these additional jets requires the calculation of a very large number of additional QCD and EW production modes.

This section describes the generators used for producing simulated Z+jets events as used in the ATLAS $H\rightarrow\tau\tau$ analysis and demonstrates its performance using the first X data at $\sqrt{s}=13$TeV as collected by the ATLAS detector. 3 possible channels are considered: Fully leptonic, one leptonic and one hadronic and fully hadronic tau decay di-tau final states. In final states containing hadronic tau decays QCD-jets that `fake' are always present meaning that a pure Z$\rightarrow\tau\tau$ final state is impossible to construct. In fully leptonic final states where the two leptons have the same flavour (ee,$\mu\mu$) the Z$\rightarrow\tau\tau$ component is suppressed in favour of a Z$\rightarrow\ell\ell$ background component. It can be assumed that the modelling of the Z boson in MC is correct and agnostic to the reconstructed final state within the statistical error of the samples produced. This allows the modelling of the Z$\rightarrow\tau\tau$ background to be assessed via and in conjunction with the Z$\rightarrow\ell\ell$ background. 

Firstly in subsection [\ref{sec:zmcsamples}] the samples considered are described. Studies into the Z lineshape between channels follows in subsection [\ref{sec:truthcomparisons}] and elucidates the assumptions into the calculation of the systematic error attributed to this background process.  The Z control region is described and prefit distributions using 5$fb^{-1}$ of data at $\sqrt{s} = 13$ TeV are shown in section [\ref{sec:zcr}]. The normalization and systematic contribution of this background into the analysis fit model and associated Nuisance Parameters (NPs) are described in section [\ref{sec:zfitmodel}]. An alternative approach using samples corresponding to theoretical scale variations is documented in section [\ref{sec:scalevariations}]. Finally the contribution of Electroweak (t-channel, VBF-like) Zjj processes and the effects of the Z and Tau polarisations are neglected in this analysis due to the investigations seen in section [\ref{sec:miscZstudies}].